\chapter{\abstractname}

%TODO: Abstract

Virtualisation is heavily used in cloud computing. Hardware resources in server farms are distributed among clients as virtual machines that are running on a hypervisor. As architecture model of software development moves more on to microservices, containerized applications arisen to allow multiple programs to run on the same virtual machine with isolation. While containerized applications increased the utility of hardware resources , they still require a full fledged operating system underneath to host them.
Running a complete operating system to run microservices is problematic in 2 ways. They allow for a greater attack surface , because a vulnerability in the underlying operating system can also compromise the running applications. This causes DevOps engineers to specify a secure host operating system for their applications, which contradicts with the "abstraction from operating system" promise of containers. Second, operating systems waste hardware resources with background jobs that are not required to the application. 

Unikernels are specialised programs that package the smallest operating system with the application itself. They don't require a running operating system to be deployed and they can be booted directly from BIOS. They have a very small attack surface as vulnerabilities can only exist within the deployed app. This reduces the code fingerprint of the deployed system greatly. They also don't waste any resources as there are no processes running other than the application. 

Because unikernels are very small operating systems , they can be booted with type-1 and type-2 hypervisors. This thesis will create a ecosystem that boots up unikernel applications scheduled by a Kubernetes cluster. The expected result is  decrease in both resources used by the cluster and time it takes to scale an application.

It will try to answer questions e.g. How to communicate with hypervisors ? How to deploy external unikernels through internet to hardware ? How to connect unikernels to kubernetes ? How does unikernels affect scheduling of kubernetes applications ?

 \textbf{Keywords}: Cloud computing, Unikernel, Kubernetes, Hypervisor
