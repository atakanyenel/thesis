\PassOptionsToPackage{table,svgnames,dvipsnames}{xcolor}

\usepackage[utf8]{inputenc}
\usepackage[T1]{fontenc}
\usepackage[sc]{mathpazo}
\usepackage[ngerman,english]{babel} % english is the same as american or USenglish
\usepackage[autostyle]{csquotes}
\usepackage[%
  backend=biber,
  url=true,
  style=numeric, % alphabetic, numeric
  sorting=none, % default == nty, https://tex.stackexchange.com/questions/51434/biblatex-citation-order
  maxnames=4,
  minnames=3,
  maxbibnames=99,
  giveninits,
  uniquename=init]{biblatex} % TODO: adapt citation style
\usepackage{graphicx}
\usepackage{scrhack} % necessary for listings package
\usepackage{listings}
\usepackage{lstautogobble}
\usepackage{tikz}
\usetikzlibrary{shapes,arrows}
\usepackage{pgfplots}
\usepackage{pgfplotstable}
\usepackage{booktabs} % for better looking table creations, but bad with vertical lines by design (package creator despises vertical lines)
\usepackage[final]{microtype}
\usepackage{caption}
\usepackage{pgf-umlsd}
\usepgflibrary{arrows} % for pgf-umlsd
\usepackage[hidelinks]{hyperref} % hidelinks removes colored boxes around references and links
\usepackage{ifthen} % for comparison of the current language and changing of the thesis layout
\usepackage{pdftexcmds} % string compare to work with all engines
\usepackage{paralist} % for condensed enumerations or lists
\usepackage{subfig} % for having figures side by side
\usepackage{siunitx} % for physical accurate units and other numerical presentations
\usepackage{multirow} % makes it possible to have bigger cells over multiple rows in a table
\usepackage{array} % different options for table cell orientation
\usepackage{makecell} % allows nice manual configuration of cells with linebreaks in \thead and \makecell with alignments
\usepackage{pdfpages} % for including multiple pages of pdfs
\usepackage{adjustbox} % can center content wider than the \textwidth
\usepackage{tablefootnote} % for footnotes in tables as \tablefootnote
\usepackage{threeparttable} % another way to add footnotes as \tablenotes with \item [x] <your footnote> after setting \tnote{x} 


% https://tex.stackexchange.com/questions/42619/x-mark-to-match-checkmark
\usepackage{amssymb}% http://ctan.org/pkg/amssymb
\usepackage{pifont}% http://ctan.org/pkg/pifont
\newcommand{\cmark}{\ding{51}}%
\newcommand{\xmark}{\ding{55}}%


\usepackage[acronym,xindy,toc]{glossaries} % TODO: include "acronym" if glossary and acronym should be separated
%\makeglossaries
\loadglsentries{pages/glossary.tex} % important update for glossaries, before document


\bibliography{bibliography}

\setkomafont{disposition}{\normalfont\bfseries} % use serif font for headings
\linespread{1.05} % adjust line spread for mathpazo font

% Add table of contents to PDF bookmarks
\BeforeTOCHead[toc]{{\cleardoublepage\pdfbookmark[0]{\contentsname}{toc}}}

% Define TUM corporate design colors
% Taken from http://portal.mytum.de/corporatedesign/index_print/vorlagen/index_farben
\definecolor{TUMBlue}{HTML}{0065BD}
\definecolor{TUMSecondaryBlue}{HTML}{005293}
\definecolor{TUMSecondaryBlue2}{HTML}{003359}
\definecolor{TUMBlack}{HTML}{000000}
\definecolor{TUMWhite}{HTML}{FFFFFF}
\definecolor{TUMDarkGray}{HTML}{333333}
\definecolor{TUMGray}{HTML}{808080}
\definecolor{TUMLightGray}{HTML}{CCCCC6}
\definecolor{TUMAccentGray}{HTML}{DAD7CB}
\definecolor{TUMAccentOrange}{HTML}{E37222}
\definecolor{TUMAccentGreen}{HTML}{A2AD00}
\definecolor{TUMAccentLightBlue}{HTML}{98C6EA}
\definecolor{TUMAccentBlue}{HTML}{64A0C8}

% Settings for pgfplots
\pgfplotsset{compat=newest}
\pgfplotsset{
  % For available color names, see http://www.latextemplates.com/svgnames-colors
  cycle list={TUMBlue\\TUMAccentOrange\\TUMAccentGreen\\TUMSecondaryBlue2\\TUMDarkGray\\},
}

% Settings for lstlistings

% Use this for basic highlighting
\lstset{%
  basicstyle=\ttfamily,
  columns=fullflexible,
  autogobble,
  keywordstyle=\bfseries\color{TUMBlue},
  stringstyle=\color{TUMAccentGreen}
}

% use this for C# highlighting
% %\setmonofont{Consolas} %to be used with XeLaTeX or LuaLaTeX
% \definecolor{bluekeywords}{rgb}{0,0,1}
% \definecolor{greencomments}{rgb}{0,0.5,0}
% \definecolor{redstrings}{rgb}{0.64,0.08,0.08}
% \definecolor{xmlcomments}{rgb}{0.5,0.5,0.5}
% \definecolor{types}{rgb}{0.17,0.57,0.68}

% \lstset{language=[Sharp]C,
% captionpos=b,
% %numbers=left, % numbering
% %numberstyle=\tiny, % small row numbers
% frame=lines, % above and underneath of listings is a line
% showspaces=false,
% showtabs=false,
% breaklines=true,
% showstringspaces=false,
% breakatwhitespace=true,
% escapeinside={(*@}{@*)},
% commentstyle=\color{greencomments},
% morekeywords={partial, var, value, get, set},
% keywordstyle=\color{bluekeywords},
% stringstyle=\color{redstrings},
% basicstyle=\ttfamily\small,
% }

% Settings for search order of pictures
\graphicspath{
    {logos/}
    {figures/}
}

% Set up hyphenation rules for the language package when mistakes happen
\babelhyphenation[english]{
an-oth-er
ex-am-ple
}

% Decide between
%\newcommand{\todo}[1]{\textbf{\textsc{\textcolor{TUMAccentOrange}{(TODO: #1)}}}} % for one paragraph, otherwise error!
%\newcommand{\done}[1]{\textit{\textsc{\textcolor{TUMAccentBlue}{(Done: #1)}}}} % for one paragraph, otherwise error!
% and
\newcommand{\todo}[1]{{\bfseries{\scshape{\color{TUMAccentOrange}[(TODO: #1)]}}}} % for multiple paragraphs
\newcommand{\done}[1]{{\itshape{\scshape{\color{TUMAccentBlue}[(Done: #1)]}}}} % for multiple paragraphs
% for error handling of intended behavior in your latex documents.

\newcommand{\tabitem}{~~\llap{\textbullet}~~}

\newcolumntype{P}[1]{>{\centering\arraybackslash}p{#1}} % for horizontal alignment with limited column width
\newcolumntype{M}[1]{>{\centering\arraybackslash}m{#1}} % for horizontal and vertical alignment with limited column width
\newcolumntype{L}[1]{>{\raggedright\arraybackslash}m{#1}} % for vertical alignment left with limited column width
\newcolumntype{R}[1]{>{\raggedleft\arraybackslash}m{#1}} % for vertical alignment right with limited column width

\begin{filecontents}{scale50.dat}
    X   Time Docker Unikernel
    1   0   0   0
    2 0.5 0 15
    3 1.0 0 19
    4 1.5 0 25
    5 2.0 0 31
    6 2.5 0 36
    7 3.0 0 40
    8 3.5 0 45
    9 4.0 0 50
    10 4.5 0 50
    11 5.0 0 50
    12 5.5 1 50
    13 6.0 3 50
    14 6.5 4 50
    15 7.0 5 50
    16 7.5 7 50
    17 8.0 12 50
    18 8.5 13 50
    19 9.0 14 50
    20 9.5 15 50
    21 10.0 16 50
    22 10.5 17 50
    23 11.0 19 50
    24 11.5 19 50
    25 12.0 21 50
    26 12.5 23 50
    27 13.0 23 50
    28 13.5 24 50
    29 14.0 25 50
    30 14.5 25 50
    31 15.0 27 50
    32 15.5 29 50
    33 16.0 29 50
    34 16.5 31 50
    35 17.0 32 50
    36 17.5 32 50
    37 18.0 34 50
    38 18.5 36 50
    39 19.0 36 50
    40 19.5 38 50
    41 20.0 39 50
    42 20.5 39 50
    43 21.0 40 50
    44 21.5 40 50
    45 22.0 43 50
    46 22.5 44 50
    47 23.0 44 50
    48 23.5 45 50
    49 24.0 46 50
    50 24.5 47 50
    51 25.0 47 50
    52 25.5 47 50
    53 26.0 48 50
    54 26.5 48 50
    55 27.0 49 50
    56 27.5 49 50
    57 28.0 50 50

\end{filecontents}

\begin{filecontents}{scale30.dat}
    Time	Docker	Unikernel
    0	0	0
    0.5	0	15
    1.0	0	19
    1.5	0	25
    2.0	0	29
    2.5	2	30
    3.0	3	30
    3.5	3	30
    4.0	6	30
    4.5	8	30
    5.0	8	30
    5.5	10	30
    6.0	10	30
    6.5	10	30
    7.0	12	30
    7.5	15	30
    8.0	15	30
    8.5	17	30
    9.0	17	30
    9.5	20	30
    10.0	20	30
    10.5	21	30
    11.0	22	30
    11.5	22	30
    12.0	23	30
    12.5	23	30
    13.0	24	30
    13.5	24	30
    14.0	24	30
    14.5	25	30
    15.0	26	30
    15.5	27	30
    16.0	27	30
    16.5	28	30
    17.0	28	30
    17.5	28	30
    18.0	29	30
    18.5	29	30
    19.0	29	30
    19.5	30	30

\end{filecontents}


\begin{filecontents}{scaledown30.dat}
    Time	Docker	Unikernel
    0	30	30
    0.5	30	17
    1.0	30	13
    1.5	30	10
    2.0	30	7
    2.5	30	2
    3.0	29	0
    3.5	29	0
    4.0	28	0
    4.5	27	0
    5.0	27	0
    5.5	27	0
    6.0	26	0
    6.5	26	0
    7.0	25	0
    7.5	24	0
    8.0	18	0
    8.5	17	0
    9.0	17	0
    9.5	17	0
    10.0	17	0
    10.5	17	0
    11.0	16	0
    11.5	15	0
    12.0	15	0
    12.5	14	0
    13.0	13	0
    13.5	12	0
    14.0	12	0
    14.5	11	0
    15.0	11	0
    15.5	10	0
    16.0	9	0
    16.5	8	0
    17.0	7	0
    17.5	6	0
    18.0	5	0
    18.5	4	0
    19.0	3	0
    19.5	2	0
    20.0	1	0
    20.5	0	0

\end{filecontents}

\begin{filecontents}{scaledown50.dat}
    Time	Docker	Unikernel
0	50	50
0.5	50	27
1.0	50	22
1.5	50	20
2.0	50	18
2.5	50	13
3.0	50	12
3.5	50	11
4.0	50	7
4.5	50	6
5.0	50	4
5.5	49	2
6.0	48	0
6.5	47	0
7.0	45	0
7.5	43	0
8.0	41	0
8.5	39	0
9.0	37	0
9.5	34	0
10.0	32	0
10.5	30	0
11.0	28	0
11.5	26	0
12.0	24	0
12.5	22	0
13.0	20	0
13.5	18	0
14.0	16	0
14.5	14	0
15.0	12	0
15.5	10	0
16.0	9	0
16.5	8	0
17.0	7	0
17.5	6	0
18.0	5	0
18.5	4	0
19.0	2	0
19.5	1	0
20.0	0	0
\end{filecontents}