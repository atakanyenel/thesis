\section{Setting up the Kubernetes Cluster}
The cloud provider selected for this thesis is Leibniz-Rechenzentrum, LRZ. They are using the OpenStack \cite{openstack} interface to provide computing services for their customers. While developing on the cloud, microK8s was used for setting up local,short-lived single-node clusters. The testing was done on a Kubernetes cluster set up by following the official documentation on their website. In that case, there is a Kubernetes master deployed on a VM with 20 GB Disk , 9 GB RAM, 2 VCPUs and running Ubuntu-18.04. The Kubernetes version is v1.16.3, which is the latest release at the time of writing. A secondary node is used for scaling experiments with the configuration similar to the master but with 4.5 GB RAM. The master node has a public IP so that devices can connect to it without a VPN. The pod network add-on used by the cluster is Cilium \cite{cilium}, as it's also being used by MicroK8s.

Another approach was to use Google Kubernetes Engine for setting up managed clusters. While their interface is easier to work with, connecting to them through non-Google-controlled devices is much more harder. They either require a device that is authorised with the \textit{gcloud} command line interface or they require a service account key file to be distributed alongside the kubeconfig file. While the first option was not a problem when registering the development PC as an additional node, it is suboptimal to install and authorise gcloud on IoT devices. It is a big application with many additional dependencies such as OpenSSL, thus unsuitable for the resource-constrained end devices. The second approach is much more resource friendly but just as similar to first approach, It is a security vulnerability waiting to be exploited as misconfigured service accounts can be used to run arbitrary commands on GCP using users' budget.

For local development, Minikube was also used on some parts of the project. Minikube only allows single node clusters unlike microK8s, but It was possible to register additional local nodes through valid kubeconfig to Minikube.
