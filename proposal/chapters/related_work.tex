% !TeX root = ../main.tex
% Add the above to each chapter to make compiling the PDF easier in some editors.

\chapter{Related Work}\label{chapter:literature}

This project relies the research going around Kubernetes and IoT at It's core. Kubernetes has a very active community around it, both in research and development, that Google Scholar returns 9.140 results when searching the technology. While kubernetes has many resources, a Google scholar search on Unikernels brings 1150 results. Most of those results focus on benchmarking unikernels with other virtualisation methods.

Using Kubernetes in an IoT environment is still being discovered. There are open source projects bringing Kubernetes experience to IoT by bridging the gap between edge devices and the cloud. The Kubeedge project \cite{kubeedge} from Huawei, runs Dockerized applications on edge devices and manages them through Kubernetes. Their future work mentions an intelligent scheduling based on data locality, network status and computing power, which this thesis aims to solve to an extend using Kubernetes labels.

Another open source solution is K3S \cite{k3s}, a lightweight kubernetes distribution designed for edge computing. K3S strips Kubernetes codebase with non-default options removed, while harcoding some options into the system. A shortcoming for K3S is that it requires a standalone setup and can not integrate with a already running non-stripped Kubernetes cluster.

A project similar to the work presented in this thesis is FLEDGE \cite{fledge}. The main difference between this thesis and FLEDGE is the location where virtual-kubelet instances are deployed and the usage of unikernels. They deploy virtual-kubelet instances in the cluster and use it as a facade between the Kubernetes environment and IoT devices. This thesis deploys them directly to IoT devices. They state in their future work that they will migrate to install virtual-kubelet agents on the end devices too, because creating a pod per device decreases the scalability of their solution.

The Fogernetes project \cite{fogernetes} from Technical University Munich proposes a labeling schema to use with Kubernetes while managing applications in fog computing. Their labeling schema categorises edge devices according to their: \textit{Location, Device Extension, Performance, Connectivity} and relies on Kubernetes scheduling algorithms to run correct applications on correct devices.

A project by Bellavista et al. \cite{Bellavista2017} proposes a Docker based approach to run applications on fog devices such as Raspberry Pi. Their solution communicates with MQTT and they use Docker Swarm as the orchestrator platform. Nevertheless, they still propose Kubernetes and Apache Mesos for alternative orchestrator methods.

In \cite{Mueller2017}, Mueller et al. compare different cloud technologies to create similar environments between layers from edge to cloud. For their future work, they propose combining lightweight technologies \textit{"to build the core of the seamless computing platform"}. This thesis shows how a lightweight technology such as Unikernel can be combined with Kubernetes, a technology that they already discuss, to help build that core.

For controlling low-resource devices, such as microcontrollers, cloud provider companies have their own solutions. E. g. Google has \textit{Google Cloud IoT Device SDK for Embedded C} repository for connecting low-end devices such as Arduino or ESP32 to their own IoT core product. It enables concurrent Pub/Sub traffic on devices using MQTT. That traffic can then be analysed on the Google Cloud where resources are abundant.

There are some research on orchestrating unikernel applications. FADES project \cite{fades} from Technical University Munich is deploying MirageOS unikernels to specialised hardware devices such as Intel NUC and Cubietruck. Their orchestrator is a custom orchestrator written in Python. An informal talk with one of the authors revealed that it's actually a good idea to use Kubernetes for orchestration.
 %Todo : can I actually say that ??

The Mikelangelo project \cite{Struckmann2018} combines Kubernetes with Unikernels. They are using a unikernel technology called OSv \cite{osv}, that wraps \textit{unmodified linux applications} with a microVM; Virtlet, a Kubernetes plugin to run virtual machines as pods and finally register unikernels as virtual machines to the plugin. Their solution even provides a unikernel registry running behing a nginx server. While their implementation is promising, the last commit to the project was almost 3 years ago and it was not possible to recreate their solution.

This thesis differs itself from the works explained above by bringing a single solution to different problems that those works try to solve. An easy-to-install agent can be used to run unikernels both on IoT devices and on hypervisors. It also operates completely without Docker on IoT devices.