\chapter{Conclusion \& Future Work}\label{chapter:conclusion}
\section{Conclusion}
This work demonstrates how different products in cloud computing environment can be meshed together to solve multiple problems at the same time. Managing IoT with Kubernetes exists, but state of the art solutions don't use it. Unikernel technology exists, but it's not being used in production. Virtual kubelet exists, but not for hypervisors. The technology presented in this thesis has the promise to improve status quo in all those areas, but it comes with a tradeoff. The developer has to sacrifice abstraction provided by the docker, and should work on hypervisor level. On the other hand, by using Kubernetes, a better abstraction is provided for IoT developers. By using this technology, a company can cherish in IoT domain using their Kubernetes knowledge. An IoT company can provide virtual kubelet providers of their products to ease adaptation in market.

It can be seen from results that unikernel is an incubating technology that just needs more time. While they are superior to docker in scalability, docker is currently \textit{good enough} for the market. The tooling around the docker is quite good and mature, with many products supporting it out of the box. Unikernels can be used for very stable, static programs that can be written in an unconventional language like OCaml. To use true potential of a unikernel, one has to use a compiled language, C++ or OCaml in most cases, or else the sealed image would be bloated with the language interpreter. That means it has a more steep learning curve when compared to languages such as Python or Nodejs.

The landscape also needs more time to develop better runtime environments like solo5. Those runtimes will eventually make unikernels accessible to different devices with less upfront work. Some \textit{making a difference} or \textit{killer app} projects are also needed to show what unikernels are capable of. A good example could be a computing project which allows hyper scalability thanks to unikernels.

It requires not much effort to assume that we are stuck with Docker for a long time until a new, currently not invented, technology comes out. Current cloud computing landscape very much supports this idea with docker being everywhere. Let's not forget, the technology behind docker , lxc was installed on our systems for a long time, and we kept on investing to virtual machines, developed orchestrators for them, built companies around them, and we were pretty sure booting up a full operating system to deploy an application was the way to go in the future. It was not. We should expect a similar jump from unikernels.

\section{Future Work}
The first improvement suggested by this project is to improve tooling around unikernel development. Extending language support is a sure way to do it as it will get attention from more developers that way. It also opens the possibility to develop more production ready programs by companies. In it's current state, it's not affordable for business oriented companies to invest in unikernels to have a significant gain.

The second improvement can come from virtual kubelet side. It's currently not possible to extend the proxy system of Kubernetes, containers created by virtual-kubelet, whether unikernel or not, can not be joined to internal Kubernetes network. This limits interoperability inside the cluster and prohibits developers from migrating their infrastructure to new runtimes. A \textit{Virtual-kube-proxy} would definitely improve the usability of their project.

A third improvement can be made to blur the line between Kubernetes and IoT. Kubernetes uses HTTPS for intercluster communication, if that system can be swapped with a more IoT friendly messaging system, such as MQTT or Kafka, without compromising any functionality or reliability of Kubernetes, resource constrainted IoT devices can connect to Kubernetes clusters with less overhead. That is not an impossible idea. Those technologies are already being used in cloud as message queues.

Another improvement regards the side product of this thesis, namely artifically created large clusters. As explained in chapter \ref{chapter:implementation}, virtual kubelet can be used to simulate large clusters in a budget friendly way. An application can be developed to mimick scaling strategies of certain cloud providers. For example, when a researcher wants to experiment with different scale up times of cloud providers, they might write an operator for their local cluster that starts up a virtual node after an arbitrary delay to simulate horizontal scaling of providers. This is a budget friendly option as they don't have to pay for actual nodes.